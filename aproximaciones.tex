\section{Aproximaciones genéricas}

\subsection{Two Pointers}

Two pointers es una técnica que se basa en mantener dos punteros en un arreglo, usualmente uno al inicio y otro al final, y moverlos de acuerdo a ciertas condiciones. Es útil para resolver problemas en los que se necesita recorrer un arreglo de manera eficiente, como por ejemplo encontrar un subarreglo con una suma específica. 

\subsubsection{Problemas en los que se puede aplicar}
\begin{enumerate}
  \item Dado un arreglo de números enteros, hayar:
  \begin{itemize}
    \item La cantidad de subarreglos con una suma específica. 
    \item El subarreglo con la suma más grande. 
    % Casos mas exoticos
    \item Bungee builder. 
    \item Encontrar la subcadena más larga que cumpla ciertas condiciones (ej., con un número limitado de caracteres únicos).

  \end{itemize}
\end{enumerate}


\subsection{Sliding Window}

\subsection{Dynamic Programming}

\subsection{Divide and Conquer}

\subsection{DFS}

\subsection{BFS}

\subsection{Binary Search}

\subsection{Backtracking}

\subsection{Greedy}

