\section{Ejercicios resueltos}

\subsection{Fuerza Bruta}
    \begin{lstlisting}[style=cpp]
    
    
    
    using namespace std;
    
    int maxSubarraySum(vector<int>& nums) {
        int n = nums.size();
        int maxSum = INT_MIN;
    
        for (int i = 0; i < n; ++i) {            // O(n)
            int currentSum = 0;
            for (int j = i; j < n; ++j) {        // O(n)
                currentSum += nums[j];
                maxSum = max(maxSum, currentSum);
            }
        }
        return maxSum;
    }
    
    int main() {
        vector<int> nums = {-2, 1, -3, 4, -1, 2, 1, -5, 4};
        cout << "Maximum Subarray Sum: " << maxSubarraySum(nums) << endl;
        return 0;
    }
    \end{lstlisting}
    \textbf{Complejidad Temporal: }
    \begin{itemize}
        \item \textbf{Peor Caso: }$O(n^2)$ (dos bucles anidados).
        \item \textbf{Mejor caso: }$O(n)$ si encuentra rapidamente una solucion optima.
    \end{itemize}

\subsection{Knapsack}
% Explicacion
\begin{itemize}
  \item \textbf{Problema: }Dado un conjunto de objetos con pesos y valores, determinar el valor maximo que se puede obtener seleccionando objetos que no excedan una capacidad $W$. 
  \item \textbf{Entrada: }Un entero $W$ $(1 \leq W \leq 10^3)$ que representa la capacidad de la mochila, seguido de $n$ enteros que representan los pesos de los objetos, y $n$ enteros que representan los valores de los objetos. 
  \item \textbf{Salida: }Imprimir el valor maximo que se puede obtener seleccionando objetos que no excedan la capacidad $W$. 

  \item \textbf{Paradigma: } Programacion Dinamica
\end{itemize}

    
    \begin{lstlisting}[style=cpp]
int knapsack(vector<int>& weights, vector<int>& values, int W) {
    int n = weights.size();
    vector<vector<int>> dp(n + 1, vector<int>(W + 1, 0));

    for (int i = 1; i <= n; ++i) {                   // O(n)
        for (int w = 1; w <= W; ++w) {               // O(W)
            if (weights[i - 1] <= w) {
                dp[i][w] = max(dp[i - 1][w], dp[i - 1][w - weights[i - 1]] + values[i - 1]);
            } else {
                dp[i][w] = dp[i - 1][w];
            }
        }
    }
    return dp[n][W];
}

int main() {
    vector<int> weights = {1, 2, 3};
    vector<int> values = {6, 10, 12};
    int W = 5;
    cout << "Maximum Value: " << knapsack(weights, values, W) << endl;
    return 0;
}

    \end{lstlisting}
    \textbf{Complejidad Temporal: }
    \begin{itemize}
        \item \textbf{Peor Caso: }$O(n *W)$, donde n es el numero de objetos y W es la capacidad de la mochila.
        \item \textbf{Espacio: }$O(n*W).$
    \end{itemize}

\subsection{Algoritmo Greedy}
    \begin{lstlisting}[style=cpp]
int activitySelection(vector<pair<int, int>>& activities) {
    sort(activities.begin(), activities.end(), [](pair<int, int> a, pair<int, int> b) {
        return a.second < b.second; // Ordenar por tiempo de finalizacion
    });

    int count = 1;
    int lastFinish = activities[0].second;

    for (int i = 1; i < activities.size(); ++i) {    // O(n)
        if (activities[i].first >= lastFinish) {
            ++count;
            lastFinish = activities[i].second;
        }
    }
    return count;
}

int main() {
    vector<pair<int, int>> activities = {{1, 3}, {2, 5}, {4, 6}, {6, 8}, {5, 7}};
    cout << "Maximum Activities: " << activitySelection(activities) << endl;
    return 0;
}
    \end{lstlisting}
    \textbf{Complejidad Temporal: }
    \begin{itemize}
        \item \textbf{Ordenar: }$O(n*logn)$, donde n es el numero de objetos y W es la capacidad de la mochila.
        \item \textbf{Seleccion: }$O(n).$
    \end{itemize}

    \subsection{Backtracking}
        \begin{lstlisting}[style=cpp]
bool isSafe(vector<string>& board, int row, int col, int n) {
    for (int i = 0; i < row; ++i) {
        if (board[i][col] == 'Q') return false;
        if (col - (row - i) >= 0 && board[i][col - (row - i)] == 'Q') return false;
        if (col + (row - i) < n && board[i][col + (row - i)] == 'Q') return false;
    }
    return true;
}

void solve(vector<string>& board, int row, int n, vector<vector<string>>& solutions) {
    if (row == n) {
        solutions.push_back(board);
        return;
    }
    for (int col = 0; col < n; ++col) {
        if (isSafe(board, row, col, n)) {
            board[row][col] = 'Q';
            solve(board, row + 1, n, solutions);
            board[row][col] = '.'; // Backtrack
        }
    }
}

vector<vector<string>> solveNQueens(int n) {
    vector<vector<string>> solutions;
    vector<string> board(n, string(n, '.'));
    solve(board, 0, n, solutions);
    return solutions;
}

int main() {
    int n = 8; // Cambiar segun el problema
    auto solutions = solveNQueens(n);
    cout << "Total Solutions: " << solutions.size() << endl;
    return 0;
}
    \end{lstlisting}
    \textbf{Complejidad Temporal: }
    \begin{itemize}
        \item \textbf{Peor Caso: }$O(N!)$, ya que explora todas las posibles configuraciones de las reinas.

    \end{itemize}


    \subsection{Dividir y Conquistar}
  

        \begin{lstlisting}[style=cpp]
int maxCrossingSum(vector<int>& nums, int l, int m, int r) {
    int leftSum = INT_MIN, rightSum = INT_MIN;
    int sum = 0;

    for (int i = m; i >= l; --i) {
        sum += nums[i];
        leftSum = max(leftSum, sum);
    }

    sum = 0;
    for (int i = m + 1; i <= r; ++i) {
        sum += nums[i];
        rightSum = max(rightSum, sum);
    }

    return leftSum + rightSum;
}

int maxSubarraySum(vector<int>& nums, int l, int r) {
    if (l == r) return nums[l];
    int m = l + (r - l) / 2;
    return max({maxSubarraySum(nums, l, m),
                maxSubarraySum(nums, m + 1, r),
                maxCrossingSum(nums, l, m, r)});
}

int main() {
    vector<int> nums = {-2, 1, -3, 4, -1, 2, 1, -5, 4};
    cout << "Maximum Subarray Sum: " << maxSubarraySum(nums, 0, nums.size() - 1) << endl;
    return 0;
}
    \end{lstlisting}
    \textbf{Complejidad Temporal: }
    \begin{itemize}
        \item \textbf{Peor Caso: }$O(n *logn)$, debido a la recurrencia $T(n) = 2T(n/2) + O(n).$

    \end{itemize}


\subsection{Monedas}

\begin{itemize}
  \item \textbf{Problema: }Dado un valor $x$ y un conjunto de monedas, determinar la cantidad minima de monedas necesarias para formar el valor $x$. 
  \item \textbf{Entrada: }Dos enteros $n$ $(1 \leq n \leq 100)$ y $x$ $(1 \leq x \leq 10^6)$, seguido de $n$ enteros que representan las monedas. 
  \item \textbf{Salida: }Imprimir la cantidad minima de monedas necesarias para formar el valor $x$. 

  \item \textbf{Paradigma: } Programacion Dinamica
\end{itemize}

\begin{lstlisting}[style=cpp]
int main() {
    int n, x;
    cin >> n >> x;

    vector<int> coins(n);
    for (int i = 0; i < n; ++i) {
        cin >> coins[i];
    }

    vector<int> dp(x + 1, INF);
    dp[0] = 0; 

    for (int i = 0; i < n; ++i) {
        for (int j = coins[i]; j <= x; ++j) {
            if (dp[j - coins[i]] != INF) {
                dp[j] = min(dp[j], dp[j - coins[i]] + 1);
            }
        }
    }

    if (dp[x] == INF) {
        cout << -1 << endl;
    } else {
        cout << dp[x] << endl;
    }

    return 0;
}
\end{lstlisting}

\subsection{Creando Strings}
\begin{itemize}
  \item \textbf{Problema: }Dado un string, imprimir todas las permutaciones posibles. 
  \item \textbf{Entrada: }Una cadena de caracteres $s$ $(1 \leq |s| \leq 8)$. 
  \item \textbf{Salida: }Imprimir todas las permutaciones posibles de la cadena $s$. 

  \item \textbf{Paradigma: } Fuerza Bruta
\end{itemize}
\begin{lstlisting}[style=cpp]
int main() {
    string input;
    cin >> input;

    sort(input.begin(), input.end());

    vector<string> permutations;
    do {
        permutations.push_back(input);
    } while (next_permutation(input.begin(), input.end()));

    cout << permutations.size() << endl;
    for (string& s : permutations) {
        cout << s << endl;
    }

    return 0;
}
\end{lstlisting}

\subsection{Subarrays}
\begin{itemize}
  \item \textbf{Problema: }Dado un arreglo de enteros, contar cuantos subarreglos suman un valor $k$. 
  \item \textbf{Entrada: }Dos enteros $n$ $(1 \leq n \leq 10^5)$ y $k$ $(1 \leq k \leq 10^9)$, seguido de $n$ enteros que representan el arreglo. 
  \item \textbf{Salida: }Imprimir la cantidad de subarreglos que suman $k$. 

  \item \textbf{Paradigma: } Dos Punteros
\end{itemize}
\begin{lstlisting}[style=cpp]
int main() {
    int largo;
    int suma;
    cin >> largo >> suma;

    vector<int> arr(largo);
    for (int i = 0; i < largo; ++i) {
        cin >> arr[i];
    }
    
    partial_sum(arr.begin(), arr.end(), arr.begin());

    int i = 0, j = 0;
    int count = 0;

    while (j < largo) {
        int currentSum = arr[j] - (i > 0 ? arr[i - 1] : 0);
        if (currentSum == suma) {
            ++count;
            ++j;
        } else if (currentSum < suma) {
            ++j;
        } else {
            ++i;
        }
    }

    cout << count << endl;
}
\end{lstlisting}
