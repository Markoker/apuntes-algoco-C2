\section{Estructura Base para Problemas de Algoritmos}

\begin{lstlisting}[style=cpp]
# include <bits/stdc++.h>

using namespace std;

typedef long long ll;
typedef pair<int, int> pii;

const int INF = 1e9;

int main() {
    ios::sync_with_stdio(0);
    cin.tie(0);
    cout.tie(0);

    // Instrucciones del problema
    return 0;
}
\end{lstlisting}


\subsection{Plantilla para Fuerza Bruta}

\begin{lstlisting}[style=cpp]
// Fuerza Bruta: Generar todas las combinaciones posibles
void fuerzaBruta(const vector<int>& nums) {
    int n = nums.size();
    for (int mask = 0; mask < (1 << n); ++mask) {
        vector<int> subset;
        for (int i = 0; i < n; ++i) {
            if (mask & (1 << i)) subset.pb(nums[i]);
        }
        // Procesar el subset generado
        for (int x : subset) cout << x << " ";
        cout << endl;
    }
}


\end{lstlisting}

\subsection{Plantilla para Programacion Dinamica}

\begin{lstlisting}[style=cpp]
// Memoizacion (Top-Down)
map<pair<int, int>, int> memo;

int DP(int i, int j) {
    if (i == 0 || j == 0) return 0; // Caso base
    if (memo.find({i, j}) != memo.end()) return memo[{i, j}];
    
    // Supongamos que el problema requiere comparar dos indices
    int result = max(DP(i - 1, j), DP(i, j - 1));
    memo[{i, j}] = result;
    return result;
}

// Tabla Iterativa (Bottom-Up)
int DP_iterative(const vector<int>& nums) {
    int n = nums.size();
    vector<vector<int>> dp(n + 1, vector<int>(n + 1, 0));
    for (int i = 1; i <= n; ++i) {
        for (int j = 1; j <= n; ++j) {
            dp[i][j] = max(dp[i - 1][j], dp[i][j - 1]);
        }
    }
    return dp[n][n];
}


\end{lstlisting}

\subsection{Plantilla para Algoritmos Greedy}

\begin{lstlisting}[style=cpp]
int greedy(vector<int>& nums) {
    sort(all(nums)); // Ordenar para facilitar decisiones
    int result = 0;
    for (int i = 0; i < nums.size(); ++i) {
        result += nums[i]; // O cualquier logica greedy
    }
    return result;
}
\end{lstlisting}

\subsection{Plantilla para Backtracking}

\begin{lstlisting}[style=cpp]
void backtracking(vector<int>& nums, vector<int>& solution, vector<bool>& used) {
    if (solution.size() == nums.size()) {
        // Procesar solucion valida
        for (int x : solution) cout << x << " ";
        cout << endl;
        return;
    }
    for (int i = 0; i < nums.size(); ++i) {
        if (!used[i]) {
            used[i] = true;
            solution.pb(nums[i]);
            backtracking(nums, solution, used);
            solution.pop_back();
            used[i] = false;
        }
    }
}



\end{lstlisting}

\subsection{Plantilla para Dividir y Conquistar}

\begin{lstlisting}[style=cpp]
int divideYConquista(const vector<int>& nums, int l, int r) {
    if (l == r) return nums[l]; // Caso base
    int mid = l + (r - l) / 2;
    int left = divideYConquista(nums, l, mid);
    int right = divideYConquista(nums, mid + 1, r);
    return max(left, right); // Combinar resultados
}
\end{lstlisting}

\section{Funciones y Atajos utiles}

Aqui se incluyen algunas funciones utiles que podrias necesitar durante el certamen.



